\section{Tuesday, September 27th}
\subsection{Detour: DT-LTI Systems \& Internal Stability}
State-Evolution Equation:
\[
    q(n+1) = A q(n) + B x(n)
\]

Output Equation:
\[
    y(n) = C q(n) + D x(n)
\]

\subsubsection{Internal Stability}
Also called \textit{Asymptotic Stability} by some.

Given
\begin{align*}
    q(n+1) = A q(n)
    \\
    y(n)=Cq(n)
\end{align*}
Zero-Input Response (ZIR). You start at $q(0)$ and just let the system go (in a circuit example, this could be seen as letting the capacitor be adapted my nature).

\begin{shaded}
The system is internally stable if $\displaystyle\lim_{n\to\infty}q(n)\to0$ regardless of the initial state ZIR $q(0)$.
\end{shaded}

Unrolling the recursion we get:
\begin{align*}
    q(n+1) &= A q(n)
    \\
    q(1) &= A q(0)
    \\
    q(2) &= A q(1) = A^2 q(0)
    \\
    &\vdots
    \\
    \Aboxed{q(n) &= A^n q(0)}
\end{align*}

If we assume $A$ has $N$ distinct eigenvectors $A\in\mathbb R^{N\times N}$ then we know:
\[
    A = V\Lambda V^{-1}
\]

Any initial state can be decomposed into a linear combination of eigenvectorsL
\begin{align*}
    \vec q(0) 
    &= \alpha_1 \vec r_1 + \cdots + \alpha_N \vec r_N = \sum_{\ell=1}^N \alpha_\ell \vec r_\ell
    \\
    \vec q(0) 
    &= \begin{bmatrix}
        | & & | & & | \\
        \vec v_1 & \cdots & \vec v_\ell & \cdots & \vec v_N \\
        | & & | & & |
    \end{bmatrix}
    \\
    \vec q(0) 
    &= V\vec \alpha
\end{align*}

\begin{align*}
    q(n) &= A^n q(0)
    \\
    A &= V\Lambda V^{-1}
    \\
    A^2
    &= V\Lambda V^{-1}V\Lambda V^{-1} = V\Lambda^2 V^{-1}
    \\
    \vdots
    \\
    A^n 
    &= V\Lambda^n V^{-1}
    \\
    &\implies
    q(n) = V\Lambda^n V^{-1} q(0)
    \\
    q(n) &= V\Lambda^n \alpha
\end{align*}
Therefore the initial state is determined by the coefficients $\alpha_i$'s.

Therefore regardless of the initial state is equivalent to saying regardless of the $\alpha_i$'s. For this to happen, we must impose some conditions on $\vec q(n)=\sum_{\ell=1}^N \alpha_\ell \lambda_\ell \vec v_\ell$

This happens if and only if $|\lambda_i|<1\ (\forall i)$. If all the eigenvalues are within the unit circle then $\displaystyle \lim_{n\to\infty} (\lambda_\ell)^n \to 0\ (\forall \ell)$.

Note that internal stability implies BIBO-stability but the converse is not true due to unobservable mode (something that cannot be excited by the input) e.g. hidden eigenvalues which are unstable.

\subsubsection{Fibonacci Example}
See HW3.4.

\subsection{CT-LTI Systems \& Frequency Response}
To be continued in next lecture -- we ran out of time today.
