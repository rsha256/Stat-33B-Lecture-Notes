\section{Tuesday, October 4th}
\subsection{Mass-Spring-Damp}
Note that you can have different matrices but they will be similar matrices (have the same eigenvalues).

\begin{align*}
    M\ddot y+D\dot y+Ky=x
    \\
    \dot q = \underbrace{\begin{bmatrix}0&1\\-\frac KM&-\frac DM\end{bmatrix}}_A \vec q+\begin{bmatrix}
        0\\
        \frac1M
    \end{bmatrix}
    x
    \\
    y = \underbrace{\begin{bmatrix}1&0\end{bmatrix}}_C \vec q
\end{align*}

\begin{align*}
    \det(\lambda I-A) = \lambda^2+\frac DM \lambda +\frac KM = 0
    \\
    \implies
    \lambda=-\frac D{2M}\pm\frac12\sqrt{\left(\frac{D}{M}\right)^2 - \frac{4K}{M}}
    \\
    \implies
    \lambda=-\frac D{2M}\pm\sqrt{\left(\frac{D}{2M}\right)^2 - \frac{K}{M}}
    \\
    \implies
    \lambda=-\frac D{2M}\pm\frac D{2M}\sqrt{1 - \frac{K}{M}\left(\frac{2M}{D}\right)^2}
\end{align*}

\hrulefill

Now we can look at cases: $D=0$
\begin{align*}
    \lambda = \pm\sqrt{-\frac KM}
    = \pm i\sqrt{\frac KM}= \pm i\omega
    \\
    \omega_0 = \sqrt{\frac{K}{M}}
    \\
    \left(\frac{D}{2M}\right)^2 - \frac KM = 0 
    \implies \lambda_1 = \lambda_2 = -\frac{D}{2M}
    \\
    \left(\frac{D}{2M}\right)^2 - \frac KM > 0 
\end{align*}
As you keep on increasing $D$, the 2 roots collide and collapse on $-\frac D{2M}$ which is a negative real number (on the negative x-axis).

\hrulefill

\begin{align*}
    \left(\frac{D}{2M}\right)^2 - \frac KM < 0 
    \\
    \implies \lambda_1, \lambda_2 \in\mathbb C
\end{align*}

\hrulefill

If we let 
\begin{align*}
    \lambda_1 = -\frac D{2M}+\frac D{2M}\sqrt{1 - \frac{K}{M}\left(\frac{2M}{D}\right)^2}
    &&\text{[This will be right of $\frac{-D}{2M}$ as we add sth $>0$ to a negative]}
    \\\lambda_2 = -\frac D{2M}-\frac D{2M}\sqrt{1 - \frac{K}{M}\left(\frac{2M}{D}\right)^2}
    &&\text{[This will be left of $-\frac{D}{2M}$ as we subtract from a negative]}
\end{align*}
This works due to the sqrt being strictly positive and less than 1 (due to being 1 minus some positive quantity).

We call it the overdamped case when we are to the left.

\hrulefill

Assuming we do not have the pathologically designed (Critically Damped) case of a single repeated eigenvalue then choosing our state vector to contain the \textbf{position} and \textbf{velocity}. Then we get
\[
    \dot q(t) = A q(t)
\]
or
\[
    q(t) = e^{At} q(0)
\]

So our initial position initial velocity are given by $q(0)$. Then 
\begin{align*}
    q(0) 
    &= 
    \alpha_1\vec v_1 + \alpha_2 \vec v_2
    \\
    \vec q(t) 
    &= 
    \alpha_1 e^{At} \vec v_1 + \alpha_2 e^{At} \vec v_2
    &&\text{[Note $e^{At}=\sum_{k=0}^\infty {t^kA^k}{k!}$]}
    \\
    &\implies e^{At}\vec v_1=e^{\lambda_1 t \vec v_1}
    &&\text{[Using the eigenfunction property]}
\end{align*}

\subsubsection{Purely Oscillatory}
\begin{align*}
    D=0\implies \lambda_1=i\omega_0, \lambda_2=-i\omega_0
    \\
    \vec q(t) = \alpha_1 e^{i\omega_0 t}\vec v_1+\alpha_2 e^{-i\omega_0 t}\vec v_2
\end{align*}

\hrulefill

Case when $\lambda_1 < 0, \lambda_2 < 0$ and $\lambda_1, \lambda_2\in\mathbb R$:

\begin{align*}
    q_1(t)
    &=\alpha_1 e^{\lambda_1 t}v_{11}+\alpha_2 e^{\lambda_2 t}v_{12}
\end{align*}

Another case is $\lambda_1=i\omega, \lambda_2=-i\omega$.

Note that complex eigenvalues come in conjugate pairs. So $\lambda = \sigma\pm i\omega_1$.
Note that $\sigma$ is a negative real scalar since eigenvalues need to be less than zero s.t. they decay.

\[
    \therefore q_1(t) = e^{\sigma t} \left[\alpha_1 e^{\lambda_1 t}v_{11}+\alpha_2 e^{\lambda_2 t}v_{12}\right]
\]

Therefore we get an oscillating decaying behavior.

\subsection{Interconnections of LTI Systems}
\subsubsection{Cascade Series}
\[
    x(t) \to \boxed{F} \to \boxed{G} \to y(t)
\]
is equivalent to 
\[
    x(t) \to \boxed{H} \to y(t)
\]
where $\boxed{H} := \boxed{\boxed{F} \to \boxed{G}}$

\begin{align*}
    h(t) &= (f \ast g)(t)
    &&\text{[As $F\to f(t)$]}
    \\
    x(n)\triangleq e^{i\omega t}
    \implies F \to F(\omega) e^{i\omega t}
    &\implies G \to G(\omega) F(\omega) e^{i\omega t}
    \\
    H(\omega) &= F(\omega) G(\omega)
\end{align*}
We have now found a property for series:
\begin{shaded}
    \[
        h(t) = (f \ast g)(t)
    \]
    \[
        H(\omega) = F(\omega) G(\omega)
    \]
\end{shaded}

\subsubsection{Parallel}
Likewise we can see that we get:
\begin{shaded}
    \begin{align*}
        h(t) &= (f + g)(t)
        \\
        H(\omega) &= F(\omega) + G(\omega)
    \end{align*}
\end{shaded}

\subsubsection{Feedback}
Given an LTI system with a feed-forward branch (Plant, $P(\omega)$) as well as a feedback branch (Controller, $K(\omega)$), find $H(\omega)$.

We know from the eigenfunction property that: $y(t)= H(\omega)e^{i\omega t}$.

Our input $x(t)\triangleq e^{i\omega t}$ tells us that:
\begin{align*}
    e^{i\omega t} + K(\omega)H(\omega)e^{i\omega t}
    \to 
    &\boxed{P(\omega)}
    \to P(\omega)\left[e^{i\omega t}
    +
    K(\omega)H(\omega)e^{i\omega t}\right]
    \\
    P(\omega)\left[1+ K(\omega)H(\omega)\right]\cancel{e^{i\omega t}}
    &=
    H(\omega)\cancel{e^{i\omega t}}
    &&\text{[$P(\omega) \to y(t)$]}
    \\
    H(\omega)
    &=
    \frac{P(\omega)}{1-K(\omega)P(\omega)}
\end{align*}

\subsubsection{Black's Formula:}
$H(\omega)=$ (Forward Gain)/(1 - Loop Gain)

\hrulefill

\[
    -60 \text{ dB}
    = 20\log_{10}\left(\frac{|\text{Out}|}{|\text{In}|}\right)
    \implies
    |\text{Out}|=\frac1{1000}|\text{In}|
\]

\hrulefill

What if we subtract $K(\omega)$ instead?


Then we get the new equation:
\[
    H(\omega)
    =
    \frac{P(\omega)}{1-(-K(\omega)P(\omega))}
    =
    \frac{P(\omega)}{1+K(\omega)P(\omega)}
    \approx
    \frac1{K(\omega)}
\]

Let $|K(\omega)P(\omega)|\gg1 
\ (\forall\omega\in[w_1, w_2])$ which is the frequency range of interest, we can then build flat $K(\omega)=K_0,$
 for $K_0<1$.
 
Note that here we assume the plant is stable -- which may not be necessarily true. If it is not the case then feedback can be used to stabilize it.

\subsection{Application 2: Inverse System Design}
I will title this after we get the main result.

Let us now reverse the roles of the controller and plant. We will also make this a ``negative feedback loop'' as we will subtract $P(\omega)$ as it goes into the adder with the input which this sum then goes through $K(\omega)$ to give us or output.

If $|K(\omega)P(\omega)|\gg1\ (\forall\omega\in[w_1, w_2])$ then:

\[
    H(\omega) = \frac{K(\omega)}{1+K(\omega)P(\omega)}
    \implies
    H(\omega) \approx \frac{\cancel{K(\omega)}}{\cancel{K(\omega)}P(\omega)}=\frac1{P(\omega)}
\]
This gives us an impulse response of $\delta$.

The title of this application is Inverse System Design.

\subsection{Application 3: Stabilization of Unstable Systems}
Given a voltage source connected to a Capacitor, we have:
\[
    y(t)=\frac1C\int_{-\infty}^t z(\tau)\mathrm d\tau
\]
\[
    h(t)=\frac1C\int_{-\infty}^t 
    \delta(\tau)\mathrm d\tau
    =\frac1C u(t)
\]

Is this absolutely integrable (e.g. $\int_{-\infty}^\infty |h(t)|\mathrm d t< \infty$)?\\
Note that the Capacitor is not BIBO-Stable.

Now with an RC circuit, realize that:
\[
    y(t)=\frac1C\int_{-\infty}^t z(\tau)\mathrm d\tau
    =\frac1C\int_{-\infty}^t \frac{x(\tau)-y(\tau)}{R}\mathrm d\tau
\]

The gain of $\frac1R$ acts like $K$ and the gain of $\frac1C$ composed with the integrator acts like $P$.

Conclusion: the RC circuit is actually \textit{a feedback system}.
